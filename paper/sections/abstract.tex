Filters with constant phase shift in conjunction with 3/6~dB
amplitude decay per octave frequently occur in sound field synthesis and
sound reinforcement applications.
%
These ideal filters, known as (half) differentiators, exhibit
zero group delay and 45/90 degree phase shift.
%
It is well known that certain group delay distortions in electro-acoustic
systems are audible for trained listeners and critical audio stimuli,
such as transient, impulse-like and square wave signals.
%
It is of interest if linear distortion by a constant phase shift is audible as
well.
%
For that, we conducted a series of ABX listening tests, diotically presenting non-phase
shifted references against their treatments with different phase shifts.
%
The experiments revealed that for the critical
square waves, this can be clearly detected, which generally
depends on the amount of constant phase.
%
Here, -90 degree (Hilbert transform) is comparably easier
to detect than other phase shifts.
%
For castanets, lowpass filtered pink-noise and percussion the detection rate
tends to guessing for most listeners, although trained listeners were able to
discriminate treatments in the first two cases based on changed
pitch, attack and roughness cues.
%
Our results motivate to apply constant phase shift filters
to ensure that also the most critical signals are
technically reproduced as best as possible.
%(cf. modern electronic dance music with typical crest factor of as low as 6 dB).
%
In the paper, we furthermore give analytical expressions for
discrete-time infinite impulse response of an arbitrary constant phase shifter
and for practical filter design.
