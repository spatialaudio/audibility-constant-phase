\section{Introduction}
\label{sec:introduction}
%
Sound reproduction systems for large audiences
are often equipped with vertical loudspeaker arrays
which shall deliver an equally pleasant acoustic experience
to the audience in terms of loudness, timbre and spatial impression.
The reproduced wavefront can be steered and shaped towards the target region
by applying delays and weights to the individual loudspeaker signals~\cite{Meyer1984a, Meyer1990}.
%
Due to the coherence of these,
the sound field typically exhibits a low-pass filter characteristic,
and thus needs to be equalized by high-pass filtering the audio input signal.
As pointed out in \cite[Ch.~3]{schultz2016diss},
the same array processing framework is used in wave field synthesis (WFS),
which has only seemingly a different goal,
namely the physical reconstruction of a desired sound field.
The loudspeaker signals for WFS are derived from
a high-frequency approximation of
the Kirchhoff-Helmholtz integral equation~\cite{spors2008revisited, zotter2013}.
%
This enables a computationally efficient implementation of WFS
which comprises of, similar to large-scale sound reproduction,
delays and weights for the individual loudspeakers
and an overall equalization filter.
%
\NewL According to the theory of WFS~\cite{spors2008revisited},
the specification of the equalization filter
depends on the geometry and shape of the loudspeaker array.
The transfer function is $i\omega$ for 3D scenarios
where 2D arrays (e.g. spherical or planar) are used.
In terms of signals and systems theory this constitutes an differentiator,
exhibiting a slope of $+6$~dB per octave
and a constant phase of $90\degree$.
For 2D scenarios using 1D arrays (e.g. circular and linear),
the filter is given as $\sqrt{i\omega}$, constituting a half-differentiator
\cite{Tseng2000,Krishna2011},
where both the slope and phase are halved to
$+3$~dB per octave and $45\degree$, respectively.
%
\NewL In practical systems, where a continuous and infinite array cannot be used,
the specification of the equalization filter has to be adjusted accordingly.
The usage of a practical array built from individual loudspeakers causes
spectral fluctuations above the so-called spatial aliasing frequency~\cite{spors2008revisited}.
Moreover, due to the finite extent of the array,
the synthesized sound field exhibits a low frequency roll off.
The high-pass filter characteristic of an ideal equalization filter
thus should be flatten out at the highest and lowest frequencies in the spectrum,
resulting in a high-pass shelving filter.
The upper limit coincides with the spatial aliasing frequency
and the lower limit is determined by the spatial extent of the array~\cite{spors2010}.
%
\NewL The digital equalization filter is typically realized either in
a finite impulse response (FIR) or
infinite impulse response (IIR) form.
FIR type equalization filters are often designed as linear phase,
while omitting the above mentioned constant phase ($90\degree$ or $45\degree$)
\cite{wierstorf2014diss, winter2019diss}.
This results in synthesized sound fields exhibiting
a negative phase shift ($-90\degree$ or $-45\degree$)
compared to the desired reference sound field
(apart from the group delay of the FIR filter).
There are also a number of IIR type equalization filters
where the constant phase spectrum is explicitly taken into account
\cite{salvador2010}
or comes as a byproduct of the minimum phase characteristics of the desired
magnitude spectrum \cite{schultz2013daga, schultz2016diss}.
The improved physical accuracy in the synthesized sound field
is well demonstrated in \cite[Fig.~9-11]{schultz2013daga}.
%
\NewL The audibility of constant phase shifts can be regarded as special issue
of the audibility of phase distortion and group delay distortion, cf.
\cite{Hansen1974_1, Hansen1974_2,Blauert1978,Suzuki1980,Lipshitz1982, Moeller2007},
often evaluated with allpass filters.
%
From these works it is known, that audibility is strongly dependent of the
signal's waveform and spectrum and the amount of the group delay in the
critical bands.
%
Generally, sensitivity for phase/group delay distortions decreases with increasing
frequency.
%
For low frequency content a different pitch and for high frequency content
ringing and different lateralization is reported for group delay distortions.
%
The polarity of highly transient signals plays a role for the audibility.
%
It was often shown, that training on phase/group delay distorted audio content
increases the sensitivity to detect them.
%
\NewL To the authors' knowledge to date,
the perceptual impact of the constant phase shift
has not been studied yet.
It is of great interest whether the existence or absence of such a phase shift
is audible, and in the special context of sound field synthesis, if this
affects the authenticity of the synthesized sound fields.
%
The paper discusses the signal processing fundamentals of discrete-time constant
phase shift in Sec.~II.
%
In Sec.~III a listening test is presented for selected audio
content and phase shifts to initially evaluate the audibility of constant phase
shifts.
%
Sec.~IV concludes the paper.
